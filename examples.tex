\section{XML jako typ danych}

\subsection{Kompatybilność}
	\begin{frame}{Kompatybilność}
		Obsługa XML 
		\begin{itemize}
			\item Microsoft SQL Server 2005 +
		\end{itemize}
	
		Środowisko testowe 
		\begin{itemize} 
			\item Microsoft SQL Server 2008 Management Studio Express
		  	\item Microsoft SQL Server Management Studio Express
		\end{itemize}
	\end{frame}

\subsection{Typ XML}
	\begin{frame}{XML w MsSQL}		
		\begin{columns}[c]
		    \column{.5\textwidth}
		    	Typ xml wprowadzony w SQL Server 2005.
		    	
		    	Użycie
		    	\begin{itemize}
					\item kolumny tabel
				  	\item zmienne
				  	\item parametry procedur
				\end{itemize}
		    \column{.5\textwidth}
		    	\includegraphics[scale=0.5]{img/type.png}\\
				\lstinputlisting[language=SQL]{code/xml_type_sample.sql}				
	    \end{columns}
	\end{frame}

\subsection{Typowany i nietypowany XML}
	\begin{frame}{Typowany i nietypowany XML (typed vs untyped XML)}		
		Typowany XML = XML + XML SCHEMA
		
		Kiedy używać nietypowanego?
		\begin{itemize}
		  \item brak schematu xml
		  \item validacja po stronie klienta
		\end{itemize}
		
		A kiedy typowanego?
		\begin{itemize}
		  \item validacja na serwerze
		  \item kwestie optymalizacyjne
		\end{itemize}
	\end{frame}
	
	\begin{frame}{Tworzenie Typed XML}
		Zdefiniowanie schematu xml		
		\lstinputlisting[language=SQL]{code/create_schema.sql}
		Przypisanie schematu do kolumny typu xml
		\includegraphics[scale=0.5]{img/schema.png}
		\lstinputlisting[language=SQL]{code/create_table.sql}
	\end{frame}

\subsection{Wyszukiwanie elementów w XML}
	\begin{frame}{XQuery i XPath}
		\begin{itemize}
		  \item XQuery - XML Query Language
		  \item XPath - XML Path Language
		  \item wykorzystywane w funkcjach query(), value(), exist(), nodes(),
		  modify()
		\end{itemize}
		Na co pozwala
		\begin{itemize}
		  \item pobieranie informacji na podstawie zadanego kryterium
		  \item filtrowanie wyników
		  \item przeszukiwanie
		  \item łączenie danych z kilku dokumentów
		  \item sortowanie, grupowanie, agregowanie
		  \item obliczenia arytmetyczne, manipulacja łańcuchami znaków
		\end{itemize}
	\end{frame}
	
	\begin{frame}{Funkcja query}	
		Funkcja do zwracanie częsci dokumentu XML.
		\begin{columns}[t]		
		    \column{.4\textwidth}	    	
		    	\begin{itemize}
				  \item wywoływana na instancji XML
				  \item jako parametr przyjmuje wyrażenie XQuery 
				  \item zwraca rezultat wykonanego wyrażenia
				\end{itemize}
		    \column{.6\textwidth}
		    	Komenda:
				\lstinputlisting[language=SQL]{code/query.sql}
				Output:	
				\lstinputlisting[language=XML]{code/query_output.xml}			
	    \end{columns}
	\end{frame}
	
	\begin{frame}{Funkcja value}
		Funkcja do pobierania wartości znacznika lub parametru, zwracanego przez
		wyrażenie XQuery
		 	
	   	\begin{itemize}
		  \item parametry funkcji to wyrażenie XQuery i typ SQL-owy 
		  \item wartość zwracaną konwertuje na typ SQL-owy z typu XSD 
		\end{itemize}
		
	   	Komenda:
		\lstinputlisting[language=SQL]{code/value.sql}
		Output:	
		\lstinputlisting[language=XML]{code/value_output.xml}	  		
	\end{frame}
	
	\begin{frame}{Funkcja exist}
		Funkcja sprawdzająca czy istnieje fragmentu xml.
		 	
	   	\begin{itemize}
		  \item parametry funkcji to wyrażenie XQuery
		  \item zwraca 1 jeżeli istnieje, i 0 przeciwnie.
		\end{itemize}
		
	   	Komenda:
		\lstinputlisting[language=SQL]{code/exist.sql}
		Output:	
		\lstinputlisting[language=XML]{code/exist_output.xml}  		
	\end{frame}
	
	\begin{frame}{Funkcja nodes}
		Funkcja przekształcająca znaczniki (nodes) na wiersze.
		 	
	   	\begin{itemize}
		  \item parametry funkcji to wyrażenie XQuery
		  \item po funkcji należy podać tabele i kolumne
		  \item zwraca znaczniki (nodes) wskazane przez wyrażenie XQuery
		\end{itemize}
		
	   	Komenda:
		\lstinputlisting[language=SQL]{code/nodes.sql}
		Output:	
		\lstinputlisting[language=XML]{code/nodes_output.xml}  		
	\end{frame}

\subsection{Modyfikacja XML}
	\begin{frame}{XML DML}
		XML DML - XML Data Modification Language
		
		Rozszerzenie XQuery o 3 słowa kluczowe 	
	   	\begin{itemize}
		  \item insert
		  \item delete
		  \item replace value of
		\end{itemize}
		
		Atrybuty, których nie można dodawać, usuwać ani modyfikować
		\begin{itemize}
		  \item dla typed XML
		  	\begin{itemize}
		  		\item xsi:nil
		  		\item xs:type
		  	\end{itemize} 
		  	\item dla untyped i typed XML
		  	\begin{itemize}
		  		\item xmlsn
		  		\item xmlns:*
		  		\item xml:base
		  	\end{itemize} 
		\end{itemize}
	\end{frame}
	
	\begin{frame}{Funkcja modify}
		Modyfikuje zawartość dokumentu XML.
		 	
	   	\begin{itemize}
		  \item parametry funkcji to wyrażenie XML DML
		  \item może zwrócić tylko error
		\end{itemize}
	\end{frame}
	
	\begin{frame}{Dodawanie elementów}
		Składnia wyrażenia insert
		\lstinputlisting[language=SQL]{code/insert_syntax.sql}
	
		Przykład
		\lstinputlisting[language=SQL]{code/insert.sql}
	\end{frame}
	
	\begin{frame}{Modyfikacja elementów}
		Składnia wyrażenia replace value of
		\lstinputlisting[language=SQL]{code/replace_syntax.sql}
	
		Przykład
		\lstinputlisting[language=SQL]{code/replace.sql}
	\end{frame}
	
	\begin{frame}{Usuwanie elementów}
		Składnia wyrażenia delete
		\lstinputlisting[language=SQL]{code/delete_syntax.sql}
	
		Przykład
		\lstinputlisting[language=SQL]{code/delete.sql}
	\end{frame}

\subsection{Indeksy xml-owe}
	\begin{frame}{XML Index}
		\begin{itemize}
		  \item Primary XML Index
		  	\begin{itemize}
		  		\item Indeksuje tagi, wartości i ścieżki w instancjach XML w kolumnie XML
			  	\item Wymaga clustered index na kluczu głównym
		  	\end{itemize}
		  \item Secondary XML Index
		  	\begin{itemize}
		  	  \item Wymaga Primary XML Index
		  	  \item Są 3 typy: PATH, VALUE, PROPERTY
		  	\end{itemize}
		\end{itemize} 
		\lstinputlisting[language=SQL]{code/indexes.sql}
	\end{frame}