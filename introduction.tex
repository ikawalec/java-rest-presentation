\subsection{What is REST?}
	\begin{frame}{REST - Representational State Transfer}
		%\begin{columns}[c]
		    %\column{.4\textwidth}
		    	\begin{itemize}
				  \item architectural style developed by W3C in parallel with HTTP/1.1
				  \item simple, stateless, client-server architecture that generally runs over HTTP 
				  \item lightweight alternative to RPC, SOAP, WSDL, CORBA
				  \item often used in mobile and web applications
				  \item server provides resources, identified by URI which can be managed by requests from clients
				\end{itemize}
		    %\column{.6\textwidth}
		    %	\lstinputlisting[language=XML]{code/xml_sample_2.xml}
	    %\end{columns}
	\end{frame}
	
\subsection{RESTful web services} 
	
	\begin{frame}{RESTful web services}	
		RESTful Web Services are web API implemented using HTTP and the principles of REST.	
		
		\vspace{1cm}
		
		It is a collection of resources which have:
		\begin{itemize}
		  \item the base URI for the web API
		  \item the internet media type of the data. (e.g., JSON, XML)
		  \item the set of operations supported by the web API using HTTP methods (e.g., GET, PUT, POST or DELETE)
		\end{itemize}
	\end{frame}	
	
	\begin{frame}{RESTful HTTP methods}	
		For collection URI - http://example.com/users
		\begin{itemize}
		  \item GET - return a list of users
		  \item PUT - replace the entire collection with another collection
		  \item POST - create a new user and generate the URI for that entry
		  \item DELETE - remove all users
		\end{itemize}
		
		\vspace{1cm}
		
		For element URI - http://example.com/users/wojciech90
		\begin{itemize}
		  \item GET - return the specified user
		  \item PUT - update the user or if it doesn't exists, create it
		  \item POST - create a new user with specified URI
		  \item DELETE - remove the addressed user
		\end{itemize}
	\end{frame}	
  
\subsection{JAX-RS API}
	\begin{frame}{What is JAX-RS?}
		JAX-RS - Java API for RESTful Web Services
	
    	\begin{itemize}
			\item it is a Java programming language API that provides support in creating web services according to REST architectural pattern
			\item uses annotations
			\item introduced in JAVA SE 5
			\item an official part of JAVA EE 6 since version 1.1
		\end{itemize}
	\end{frame}
	
	\begin{frame}{JAX-RS specification}
		JAX-RS provides annotations to map resource class as a web resource.
		\begin{itemize}
		  \item @Path - specified the relative path for a resource class or method
		  \item @GET, @PUT, @POST, @DELETE - speficy the HTTP request type of a resource
		  \item @Produces - specifies the response internet media types
		  \item @Consumes - specifies the accepted request internet media types
		\end{itemize}	
	\end{frame}
	
	\begin{frame}{JAX-RS annotations for methods}
		Annotation for method parameters:
		\begin{itemize}
		  \item @PathParam - binds parameter to a path segment
		  \item @QueryParam - value of an HTTP query parameter
		  \item @HeaderParam - value of an HTTP header
		  \item @CookieParam - value of a cookie
		  \item @FormParam - value of a form
		  \item @DefaultValue - speficies a default value for above bindings when the key is not found
		  \item @Context - returns the entire context of the object (for instance HttpServlerRequest) 
		\end{itemize}
	\end{frame}
		
	\begin{frame}{JAX-RS Example}
		\lstinputlisting[language=JAVA]{code/UserService.java}
	\end{frame}	
		
	\begin{frame}{Implementations of JAX-RS}
		\begin{itemize}
			\item Jersey - the reference implementation from Oracle
			\item Apache CXF - open source web service framework 
			\item Restlet - created by Jerome Louvel, a pioneer in REST frameworks
			\item RESTeasy - Jboss's implementation
			\item and more
		\end{itemize}
	\end{frame}
